\section{Introduction}

\section{Présentation des problèmes}

\subsection{SSCFLP}

Le problème SSCFLP (Single Source Capacited Facility Location Problem) consiste à choisir selon une liste prédéfinie des emplacements pour ouvrir des services qui desservent des clients. Les services ont une capacité maximale et les demandes des clients doivent être satisfaites. Chaque client doit être lié a un unique service et l'objectif est de minimiser le  coût d'ouverture des services et le coût de connexion d'un service à un client. \newline

n clients et m services
$w_i$ : les demandes des clients
$s_i$ : les capacités des services

Sous forme de programme linéaire :


\begin{align*}
x_{ij} &= \begin{cases}
        1 \text{ si le client j est affecté au service i}\\
        0 \text{ sinon} \\
    \end{cases}
&&i \in \{1..m\}, j \in \{1,..,n\} \\
y_i &= \begin{cases}
        1 \text{ si le service i est ouvert}\\
        0 \text{ sinon} \\
    \end{cases}
&&i \in \{1,..,m\}
\end{align*}



\begin{align}
min z = &\sum\limits_{i=1}^m \sum\limits_{j=1}^n x_{ij} + \sum\limits_{i=1}^m y_i \\
  s/c \quad &\sum\limits_{j=1}^n w_j x_{ij} \leq s_i y_i \quad \forall i \in \{1, .., m\} \\
            &\sum\limits_{i=1}^m x_{ij} = 1 \quad \forall j \in \{1, .., n\} \\
            &x_{ij} \in \{0,1\}, y_i \in \{0,1\} \quad \forall i \in \{1,..,m\}, \forall j \in \{1,..,n\}
\end{align}

\subsection{CFLP}

Dans le problème CFLP (Capacited Facility Location Problem), les clients peuvent être reliés à plusieurs services. Dans cette variante, les variable $x_{ij}$ ne sont plus binaires. La demande d'un client est donc fractionnée entre les différents services auxquels il est connecté. La fonction objectif et la première contrainte ne changent pas. la deuxième contrainte devient :
$ \sum \limits_{i=1}^n x_{ij} \geq 1 \quad \forall j \in \{1,..,m\}$

\section {UFLP}

Dans le problème UFLP (Uncapacited Facility Location Problem), les clients peuvent encore une fois être reliés à plus d'un service. De plus, il n'y a plus de contrainte de capacité.
La deuxième contrainte du SSCFLP devient est remplacée par la contrainte du CFLP.
La première contrainte est retirée et est remplacée par
$ x_{ij} \leq y_i \quad \forall j \in \{1,..,m\}$. De ce fait, si au moins un client est relié au service i, alors ce service sera indiqué comme étant ouvert.


\section{Choix des instances}

Pour débuter le projet, il est nécessaire de choisir de petites instances.
Pour commencer, certaines des plus petites instances ont été choisies.
Parmis les instances de Elena Fernández Aréizaga, les instances p1 et p4 ont été choisies. Ce sont des instances qui contiennent 20 clients et 10 services.
Dans ce même jeu d'instance, les instances p28 et p31 ont été choisies. Ces instances continennet 50 clients et 20 services. \newline

Des instances de tailles moyennes ont été choisies parmis celles de Beasley. Ce sont des instances comprenant entre 16 et 50 services et 50 clients.
Ces instance permettent donc de comparer l'effcicacité des méthode de résolution lorsque le nombre de service se rapproche du nombre de client.
Parmis les instances de Beasley, 2 instances avec large coefficients ont également été sélectionnées les instances capa1 et capb1 qui ont décrivent 100 services et 1000 clients. \newline

Enfin, des instances de très grande taille ont été choisies parmis les instances de Yang : une instance de 60 services et 300 clients et une instance de 80 services et 400 clients.
Ces instances sont fournis avec les solutions exactes, ce qui permet de comparer les meilleures valeurs retournées par le solver dans le cas où il ne termine pas en temps raisonable. \newline

Un unique parser a été fait dans le programme pour charger des instances au format OR LIBRARY seulement.
Etant donné que les instances de Areizaga ne sont pas dans le même format, un petit convertisseur écrit en julia a été réalisé. Le programme ne traite donc que des instances au format OR LIBRARY.
