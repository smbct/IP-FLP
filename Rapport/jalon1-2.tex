\section{Introduction}

Les problèmes de FLP (Facility Location Problem) sont de sproblèmes très étudiés en programmation linéaire.
L'étude de ces problèmes a un intérêt pour la résolution de plusieurs problèmes en nombres entiers.
Dans cette partie du TP, les problèmes sont formulés et des instances sont choisies pour tester les algorithmes à implémenter.
Enfin, une implémentation de la résolution avec un solver générique est réalisée et les temps obtenus pour les instances sont proposés.

\section{Présentation des problèmes}

\subsection{SSCFLP}

Le problème SSCFLP (Single Source Capacited Facility Location Problem) consiste à choisir, selon une liste prédéfinie, des emplacements pour ouvrir des services qui satisfont des clients. Les services ont une capacité maximale et les demandes des clients doivent être satisfaites. Chaque client doit être lié a un et un unique service et l'objectif est de minimiser le coût d'ouverture des services et le coût de connexion d'un service à un client. \newline

On note n le nombre de client à satisfaire et m le nombre de service que l'on peut ouvrir.
On note également $w_i$, $i \in \{1,..,m\}$ les demandes des clients et $s_j$, $j \in \{1,..,n\}$ les capacités des services.
Enfin on note $f_i$, $i \in \{1,..,m\}$ les coûts d'ouverture des services et $c_{ij}$, $i \in \{1,..,m\}$, $j \in \{1,..,n\}$ les coûts de connexion du service i au service j. \newline

Le problème peut alors s'écrire sous la forme de programme linéaire :

\begin{align*}
x_{ij} &= \begin{cases}
        1 \text{ si le client j est affecté au service i}\\
        0 \text{ sinon} \\
    \end{cases}
&&i \in \{1..m\}, j \in \{1,..,n\} \\
y_i &= \begin{cases}
        1 \text{ si le service i est ouvert}\\
        0 \text{ sinon} \\
    \end{cases}
&&i \in \{1,..,m\}
\end{align*}



\begin{align}
min \quad z = &\sum\limits_{i=1}^m \sum\limits_{j=1}^n x_{ij}c_{ij} + \sum\limits_{i=1}^m y_i f_i \\
  s/c \quad &\sum\limits_{j=1}^n w_j x_{ij} \leq s_i y_i \quad \forall i \in \{1, .., m\} \\
            &\sum\limits_{i=1}^m x_{ij} = 1 \quad \forall j \in \{1, .., n\} \\
            &x_{ij} \in \{0,1\}, y_i \in \{0,1\} \quad \forall i \in \{1,..,m\}, \forall j \in \{1,..,n\}
\end{align}

\subsection{CFLP}

Dans le problème CFLP (Capacited Facility Location Problem), les clients peuvent être reliés à plusieurs services. Dans cette variante, les variable $x_{ij}$ ne sont plus binaires. La demande d'un client est donc fractionnée entre les différents services auxquels il est connecté. La fonction objectif et la première contrainte ne changent pas. La deuxième contrainte devient :
$ \sum \limits_{i=1}^m x_{ij} \geq 1 \quad \forall j \in \{1,..,n\}$.
Enfin on a $x_{ij} \in [0,1]$.

\subsection {UFLP}

Dans le problème UFLP (Uncapacited Facility Location Problem), les clients peuvent encore une fois être reliés à plus d'un service. De plus, il n'y a plus de contrainte de capacité.
La deuxième contrainte du SSCFLP est toujours remplacée par la deuxième contrainte du programme de CFLP.
La première contrainte est retirée et est remplacée par
$ x_{ij} \leq y_i \quad \forall i \in \{1,..,m\}, j \in \{1,..,n\}$.
De ce fait, si au moins un client est relié au service i, alors ce service sera indiqué comme étant ouvert.
Enfin, on a à nouveau $x_{ij} \in [0,1]$.


\section{Choix des instances}

Pour débuter le projet, il est nécessaire de choisir de petites instances.
Parmis les instances de Holmberg, les instances p14, p15, p20, p22 et p23 ont été choisies.
Ce sont des instances qui contiennent 50 clients et 20 services. \newline

Des instances de tailles similaires ont été choisies parmis celles de Beasley.
Les instances retenues sont cap64 et cap73 qui ont une taille de 16 services et 50 clients, cap94 et cap103 qui ont une taille de 25 et 50 et enfin cap122 qui a une taille de 50 et 50.\newline

Un unique parser a été fait dans le programme pour charger des instances au format OR LIBRARY seulement.

\section{Solver générique}

\subsection{Solver utilisé}

C'est le solver glpk qui a été utilisé, interfacé en c, pour résoudre les instances de SSCFLP.

\subsection{Temps obtenus}

Les temps ci-dessous proviennent de l'exécution de la résolution sur une même machine.
Le jeu auquel appartient l'instance est précisé à chaque fois.
La taille représente, dans l'ordre, le nombre de service ouvrable et le nombre de client à satisfaire.
Enfin, tous les temps sont indiqués en millisecondes.
Lorsque la résolution ne s'est pas terminée après deux minutes, la meilleure valeur obtenue est indiquée, suivie de *. \newline

\begin{tabular}{|c|c|c|c|}
    \hline
    instance  & taille &  valeur obtenue & temps (ms) \\
    \hline
    Beasley cap103.dat & 25 * 50 & 895592.7625* & 119527.131 \\
    Beasley cap122.dat & 50 * 50 & 871199.7375* & 119920.451 \\
    Beasley cap64.dat & 16 * 50 & 1053197.4375 & 462.846 \\
    Beasley cap73.dat & 16 * 50 & 1010641.45 & 1941.671 \\
    Beasley cap94.dat & 25 * 50 & 950608.425 & 63156.697 \\
    Holmberg p14.dat & 20 * 50 & 7137 & 8487.26 \\
    Holmberg p15.dat & 20 * 50 & 8808 & 3881.686 \\
    Holmberg p20.dat & 20 * 50 & 10486 & 5350.699 \\
    Holmberg p22.dat & 20 * 50 & 7092 & 10722.105 \\
    Holmberg p23.dat & 20 * 50 & 8746 & 5791.466 \\
    \hline
\end{tabular}
