\section{Introduction}

Dans cette partie, un branch and bound est mis en oeuvre pour résoudre le problème de SSCFLP.
Il fait appel à glpk pour la relaxation continue.

\section{Méthodes de branchement}

\subsection{Variable de branchement}

Il a d'abord été testé de brancher sur les services en premier lieu, et sur les clients en deuxième.
Le branchement sur un service permet ensuite d'induire un certain nombre d'affectation pour les clients.
En effet, un service affecté à 0 implique que tous les clients ne pourront pas lui être connecté et donc les variables correspondants à ces clients auront une valeur de 0. \newline

Dans la modélisation du problème, un client n'est affecté qu'à un seul service. Du coup, pas besoin de garder en mémoire toutes les variables de connexions.
Seulement un tableau des variables affectées et un tableau de service auxquels un client est affecté.
